% abstract.tex: Abstract

An active fluid denotes a suspension of particles, cells and macromolecules that are capable of transducing free energy into systematic motions. Converting energy at individual constituent scales, these systems are constantly driven out of equilibrium and display unusual phenomena, including a transition to a zero viscosity superfluid-like state and a transition to a collective moving turbulent state. These curious transitions are direct consequences of the motions of active particles, and are absent from classical complex fluids without self-propulsion. By elucidating the causes and consequences these phenomena, we will not only expand the knowledge of complex fluids, but also provide deeper understandings on the biological and ecological impact of living organism behavior.

In this thesis, experimental investigations on the rheology of active fluids is presented. Specifically, the viscosity of bacterial suspensions is significantly reduced by confining walls. We show that this effect results from upstream swimming bacteria near the confining walls, which collectively exert stress on the fluids.

The collective motions in dense bacterial suspensions are investigated. In particular, we present the first experimental study on the giant number fluctuation - a landmark of collectively moving active particles - in 3-dimensional bacterial suspensions. Our measurement agrees in part with theoretical predictions: measuremnt is consistent with theory at low and high concentrations, but displays a curious deviation from the theories at intermediate concentrations. We show that the deviation results from a strong interplay between the flow induced by bacterial motions and giant number fluctuations, which is absent in existing theories.

In addition, we measure the critical conditions of the transition from disordered state to turbulent state in bacterial suspensions. We present the experimental results in a phase diagram, serving as a benchmark for existing and future theories. We put forward a heuristic model based on two-body hydrodynamic interactions, hoping to understand the transition in a more intuitive way and to stimulate theoretical advancement.

Our experiments allow quantitative understanding of active fluids and lay the foundation of applying active fluids to real world challenges.
