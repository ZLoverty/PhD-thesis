\chapter[The Emergence of Active Turbulence]{The Emergence of Active Turbulence\footnote[1]{
Reproduced in part with permission from (Yi Peng, Zhengyang Liu and Xiang Cheng, ``Imaging the emergence of bacterial turbulence using light-powered Escherichia coli'', \textit{arXiv e-print}).
}}
\label{the-emergence-of-active-turbulence}
%%%%%%%%%%%%%%%%%%%%%%%%%%%%%%%%%%%%%%%%%%%%%%%%%%%%%

\section{Introduction}
Collective motions of biological systems such as bird flocks, fish schools and bacterial swarms are the most vivid examples of the emergent behaviors of active matter \cite{Marchetti2013}. While moving independently at low density, self-propelled units in active matter can move collectively at high density, giving rise to coherent flows at length scales much larger than the size of individual units. In bacterial suspensions, these coherent flows exhibit a chaotic pattern of intermittent vortices and jets, reminiscent of turbulent flows at high Reynolds numbers. Hence, the flows induced by bacterial collective swimming are also known as active or bacterial turbulence \cite{Wolgemuth2008, Wensink2012, Linkmann2019, Dunkel2013}.
Extensive theoretical and numerical studies have been conducted in understanding the physical principles underlying the nonequilibrium transition between the disordered and the turbulent states in bacterial suspensions \cite{Marchetti2013, Ramaswamy2010, Koch2011, Saintillan2015}.
Particularly, kinetic theories have been developed by extending the classic models of suspensions of passive rod-shaped particles \cite{Koch2011, Saintillan2015, Stenhammar2017}. The theories consider the probability distribution of the position and orientation of bacteria based on a Smoluchowski equation, where a bacterium is modeled as a rigid rod exerting a pusher-type force dipole on the suspending fluid. In addition to self-propulsion, the motion of bacteria is further coupled with a mean-field background Stokes flow, driven by an average active dipolar stress as well as more conventional particle-induced viscous stresses.
Using the kinetic theories, Saintillan and Shelley first showed that a long wavelength hydrodynamic instability drives the transition to active turbulence in suspensions of pusher swimmers \cite{Saintillan2015, Saintillan2008a, Saintillan2008b, Hohenegger2010}.
The transition kinetics was further studied beyond the linear regime by numerically solving kinetic equations \cite{Saintillan2008b} and simulations of self-propelled rods \cite{Saintillan2012}.
By incorporating the effect of tumbling and rotational diffusion in the kinetic theories, Subramanian and Koch independently showed that the isotropic disordered state of bacterial suspensions becomes unstable above a threshold concentration in the bulk limit \cite{Koch2011, Subramanian2009}. The threshold concentration increases with the random reorientation and decreases with the swimming activity of bacteria. In general, the kinetic theories assume long-range hydrodynamic dipolar interactions between bacteria and are therefore strictly valid only for dilute suspensions. Nevertheless, the onset of bacterial collective swimming is typically observed at intermediate to high concentrations where steric interactions are supposed to be important \cite{Aranson2007, Ezhilan2013, Cisneros2011}, raising the question on the relative importance of hydrodynamic versus steric interactions in the formation of bacterial turbulence.
Indeed, simulations and models with steric interactions have also been developed, which successfully predicted the rise of collective motions both with hydrodynamic interactions \cite{Aranson2007, Ezhilan2013} and without \cite{Sambelashvili2007, Baskaran2010}.
Although steric interactions have been shown to play a leading role in bacterial collective swimming in 2D or quasi-2D systems such as Hele-Shaw cells with two confining walls \cite{Ishikawa2006, Nishiguchi2017} and bacterial monolayers on agar substrates \cite{Darnton2010, Zhang2010}, it is still far from clear what is the dominant interaction leading to bacterial turbulence in 3D suspensions.

In comparison with the theoretical development, definitive experiments on bacterial suspensions that can quantitatively verify theoretical predictions are still few and far between \cite{Koch2011, Saintillan2015}.
Pioneering experiments on suspensions of Bacillus subtilis first showed the emergence of bacterial turbulence at high concentrations in absence of bioconvection and chemotaxis \cite{Cisneros2011, Dombrowski2004, Cisneros2007, Sokolov2007, Sokolov2009}.
Various physical properties of bacterial turbulence such as density fluctuations \cite{Sokolov2009}, coherent and defect flow structures \cite{Sokolov2012, Ryan2013, Guo2018, Li2019} and mass and momentum transports \cite{Peng2016, Lopez2015, Liu2019, Yang2016} have been subsequently studied.
While most of these studies focused on the rise of bacterial turbulence with increasing bacterial concentrations, few experiments have considered other factors important for bacterial turbulence such as the swimming speed of bacteria \cite{Sokolov2012, Ryan2013} and the presence of defective immobile bacteria. A systematic mapping of the phase diagram of 3D bacterial flows over a large control parameter space is still lacking. Such an experimental phase diagram is essential to verify the quantitative prediction of the kinetic theories. Particularly, the study of the effect of doping immobile bacteria on the phase dynamics will provide strong evidence to resolve the controversy over the dominant interaction responsible for collective swimming in 3D suspensions of microorganisms \cite{Aranson2007, Ezhilan2013}. Increasing the ratio of immobile to mobile bacteria in a suspension of fixed total bacterial concentration would proportionally reduce the active stress and thus suppress the hydrodynamics-driven collective motion. On
the contrary, adding immobile bacteria to a suspension with a fixed concentration of mobile bacteria would promote steric interactions due to the rigid rod-shaped body of immobile bacteria and therefore enhance the collective motion if steric interactions are the leading cause of bacterial turbulence.

Moreover, the kinetic pathway towards the collective motion of swimming microorganisms remains largely unexplored. The kinetics of a phase transition affects both the transition rate and the structure of intermediate states and is of importance in understanding the phase dynamics of equilibrium systems \cite{Peng2015}. It plays a key role in distinguishing first-order and second-order phase transitions. Kinetics is equally important in a nonequilibrium phase transition, which can reveal not only the nature of the transition but also the missing link between the rise of the macroscopic order and the microscopic interparticle interaction. Hence, resolving the route to bacterial turbulence---a premier example of collective motions---addresses the central question in active matter: how do random self-propelled units self-organize into large-scale dynamic structures?

Here, we aim to address the above open questions. Our experiments provide not only the most comprehensive phase diagram of the flow of 3D bacterial suspensions heretofore, but also the detailed characterization of the kinetic route of the bacterial turbulent transition. Our experimental phase diagram quantitatively agrees with the kinetic theories, as well as our simple hydrodynamic model balancing dipolar interactions and rotational diffusion of bacteria. By further examining the effect of doping immobile bacteria on the phase dynamics, we show unambiguously that hydrodynamic interactions dominate the formation of collective swimming in 3D bacterial suspensions. Moreover, our kinetic measurements reveal an unexpected kinetic pathway near the phase boundary in analogy of the nucleation and growth process in equilibrium phase transitions and confirm the existence of a long-wavelength instability deep inside the turbulent phase. Together, our experiments validate the basic assumption of the kinetic theories and corroborate the principal predictions of the theories on the transition point and the mode of instability. Furthermore, our study uncovers new kinetic features of the bacterial turbulent transition and sheds new
light on the nature of the nonequilibrium phase transition in active microbiological systems. These findings open new questions for future experimental and theoretical development.
\section{Methods}

\section{Results}

\section{Discussion and Conclusion}
