\chapter{Summary and Outlook}
\label{summary-and-outlook}
%%%%%%%%%%%%%%%%%%%%%%%%%%%%%%%%%%%%%%%%%%%%%%%%%%%%%
\section{Summary}

In this thesis, I described my investigations on the rheological properties of active fluids and the collective motions of the active swimmers, using \textit{E. coli} suspensions as a model system. Due to self-propulsion, active fluids exhibit unusual interplay with system boundaries, and self-organize into complicated patterns, which are absent in equilibrium systems.

We used a microfluidic viscometer to study the confinement effect on the viscosity of active fluids. We showed that such a confinement effect arises from the interplay between bacterial motility and confining surfaces. We found that the upstream swimming bacteria near the boundaries played a key role in the confinement effect on the viscosity of bacterial suspensions. We developed a simple model based on this observation, which showed a qualitative agreement with our rheological and microscopic dynamics measurements.

We further studied active turbulence---arguably the most intriguing phenomenon in active fluids---using a light-powered \textit{E. coli} mutant strain. Our systematic experiments explored a large parameter space and identified the phase boundaries between disordered and ordered states. We also uncovered the kinetics of the transition into active turbulence, which strongly supported the predictions of the kinetic theories.

We also studied the giant number fluctuations and the energy spectra are systematically studied. Both measurements provided supporting evidence for several  predictions of the kinetic theories and other models. Our experimental results also revealed a strong coupling between GNF and energy spectra, deepening the understanding of the microscopic origin of the GNF, the landmark phenomenon of active fluids.

We hope that the systematic measurements can stimulate further advancement of active fluid research.

\section{Outlook}

\textit{E. coli} bacteria have served as a great model swimmer for active fluid experiment, for their robustness and ease to cultivate. However, assuming they are just reckless swimmers is sometimes not appropriate, since they are ultimately microorganisms and are certainly more complex than what we expect. We describe their swimming as ``run-and-tumble'' fashion, but in fact, their motion can be highly correlated with the environment, which is affected by chemotaxis, gyrotaxis, rheotaxis, etc.. We have always been looking at these bacteria from a pure physical point of view, and have been encouraged that some seemingly biological processes, such as the collective motions, are indeed governed by physical principles. In other times when physical principles don't capture a phenomenon, however, we will have to incorporate the biological aspects of them.

Due to the above reasons, I hope that in the future investigations on biological active fluids, the biological aspects of the swimmers can be properly accounted for. In such a way, we may get closer to the ultimate goal of active matter study --- to better understand the biological and ecological impact of animal behavior.

For pure physics study, I hope that the rapid growth of artificial swimmers, such as chemical reaction driven Janus particles and electric field driven Quincke rollers, will one day replace the use of biological swimmers since they are much simpler than \textit{E. coli} and thus behave in a less complex way.

At the end of the thesis, I wish that these two directions can bring us closer to the ultimate goals of active fluid research: to understand biology better, and to have real world applications.
