%%%%%%%%%%%%%%%%%%%%%%%%%%%%%%%%%%%%%%%%%%%%%%%%%%%%%%%%%%%%%%%%%%%%%%%%%%%%%%%%
% intro.tex: Introduction to the thesis
%%%%%%%%%%%%%%%%%%%%%%%%%%%%%%%%%%%%%%%%%%%%%%%%%%%%%%%%%%%%%%%%%%%%%%%%%%%%%%%%
% Outline:
% - Active fluids
% - Emergent properties
% - Rheology: historical point of view
% - Collective motion: historical point of view
%%%%%%%%%%%%%%%%%%%%%%%%%%%%%%%%%%%%%%%%%%%%%%%%%%%%%%%%%%%%%%%%%%%%%%%%%%%%%%%%




\chapter{Introduction}
\label{intro_chapter}
%%%%%%%%%%%%%%%%%%%%%%%%%%%%%%%%%%%%%%%%%%%%%%%%%%%%%%%%%%%%%%%%%%%%%%%%%%%%%%%%

\begin{itemize}

\item Chapter 1 briefly describe the history and significance of active fluid research.

\item Chapter 2 presents the experimental techniques used in this theis.

\item Chapter 3 talks about one of the emergent properties: reduced viscosity. Large portion of  this chapter have been published in \cite{Liu2019}.

\item Chapter 4 talks about another emergent property: giant number fluctuation. This work is under preparation for submission.

\item Chapter 5 presents the study on the transition from disordered state to active turbulence in light-powered bacterial suspensions. This work is conducted with a close collaboration with Yi Peng and Xiang Cheng. Large portions of this chapter has been published in \cite{Peng2020}. Yi Peng, Zhengyang Liu and Xiang Cheng conceived the experiment. Zhengyang Liu constructed the light-powered bacteria. Yi Peng performed the experiment. Zhengyang Liu and Yi Peng did the data analysis. All authors contribute to the model development and writing of the manuscript.

\item Chapter 6 summarizes the contributions of this thesis and provides the outlook on future research.

\item Appendix A shows details of the construction of light-powered \textit{E. coli}.

\item Appendix B provides details of several particle tracking tools I developed.

\item Appendix C shows details of photolithography.

\end{itemize}
%%%%%%%%%%%%%%%%%%%%%%%%%%%%%%%%%%%%%%%%%%%%%%%%%%%%%%%%%%%%%%%%%%%%%%%%%%%%%%%%


%%%%%%%%%%%%%%%%%%%%%%%%%%%%%%%%%%%%%%%%%%%%%%%%%%%%%%%%%%%%%%%%%%%%%%%%%%%%%%%%
\section{Active Fluids}
\label{active-fluids}
Active fluids are suspensions of active agents such as cells, particles and biological macromolecules that are capable of utilizing chemical energy to sustain their self-propulsion. The concept roots from a broader class of matter: soft matter, which includes polymers, surfactants and colloidal grains \cite{DeGennes1992}. Active fluids, like soft matter, are known for their complexity and flexibility. The unique self-propulsion has endowed active fluids with more intrguing and counter-intuitive properties, especially the emergent self-organized collective phenomena\cite{Glotzer2015}. The research on collective phenomena dates back to the 1980s, when flocking birds, schooling fish, herding beasts and even human crowds (Fig.~\ref{fig:1-1}a-c) were regarded as an orientationally ordered phase of living matter, in analogy with ferromagnetic spins
\cite{Reynolds1987, Vicsek1995, Narayan2007, Ward2008, Ballerini2008, Silverberg2013}. This idea has since evoked enormous research interest. More recently, besides macroscopic systems, smaller and more laboratory accessible model systems have joined. As examples, actin filaments powered by motor proteins and bacteria exhibit turbulence-like swirling patterns, and synthetic active colloids form dynamic clusters
\cite{Dunkel2013, Wensink2012, Buttinoni2013, Palacci2013, Sanchez2012, Schaller2010, Sokolov2007}. Observations have revealed that most patterns of collective motion are universal in different systems. As of today, these universal patterns can be qualitatively reproduced by simple models with collision rules and noise. And quantitative description is developing with more observations available, which is bound to have impactful applications, including understanding the reaction of panic crowd and predicting the migration of fish schools
\cite{Vicsek2012}.

\begin{figure}[!htbp]
	\begin{center}
	\includegraphics[width=5.5 in]{Figs/1-Intro/1.pdf}
	%select pdftexify command to run jpg or pdf files
	\end{center}
	\caption[Figure 1.1: Examples of living matters and active fluids.]
	{
  (a) Flocking birds, (b) people in a mosh pit at heavy metal converts, (c) schooling fish, (d) herding sheeps, (e) swarming bacteria (f) microtubule and (g) clustering active Janus particles.
  Scalebars in (e) and (g) are 10 \textmu m. Scalebar in (f) is 200 \textmu m. Images courtesy of Robert Wolstenhome (a), Ulrike Biets (b) \cite{Silverberg2013}, biographic (c, d), DeCamp (f) \cite{DeCamp2015} and Palacci (g) \cite{Palacci2013}.
	}
	\label{}
\end{figure}






\section{Novel Properties}
\label{emergent-properties}
Active fluids exhibit novel properties such as enhanced diffusion and reduced viscosity \cite{Ramaswamy2010}. The enhanced diffusion is attributed to the interaction - steric collision or hydrodynamic perturbation - between tracer particles and active agents (swimmers)
\cite{Wu2000, Peng2016, Caspi2000, Morozov2014, Patteson2016, Leptos2009,
 Yang2016, Valeriani2011, Kurtuldu2011}.
And the reduced viscosity is explained by hydrodynamic stress and swim stress \cite{}. In this section, the existing works regarding these novel properties are reviewed, and motivations for investigating the rheology of bacterial suspensions under confinement (Sec.~\ref{rheology-of-bacterial-suspensions-under-confinement}) will be discussed.


\section{Rheology}
Polystyrene spherical tracer particles, typically a few microns across, show higher diffusivity in bacterial suspensions and active microtubules \cite{Wu2000, Caspi2000}.

\section{Collective motion}
