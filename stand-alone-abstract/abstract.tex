% abstract.tex: Abstract
\documentclass[11pt]{article}

\begin{document}

\begin{center}
  \large \fontfamily{qhv}\selectfont
\textbf{Novel properties and \\ emergent collective phenomena of active fluids}
\end{center}


An active fluid denotes a suspension of particles, cells and macromolecules that are capable of transducing free energy into systematic motions. Converting energy at individual constituent scales, these systems are constantly driven out of equilibrium and display unusual phenomena, including a transition to a zero viscosity superfluid-like state and a transition to a collective moving turbulent state. These curious transitions are consequences of the self-propulsion of active particles, and are absent in classical complex fluids without spontaneous motions.

In this thesis, we presnet an experimental investigation on the rheology of active fluids under confinement. Specifically, the viscosity of bacterial suspensions is significantly reduced by confining walls. We show that this effect results from upstream swimming bacteria near the confining walls, which collectively exert stress on the fluids and push the fluids to flow. A phenomenological model is proposed which qualitatively captures the confinement effect on the viscosity of bacterial suspensions.

The collective motions in dense bacterial suspensions are investigated. In particular, we present the first experimental study on the giant number fluctuation - a landmark of collectively moving active particles - in 3-dimensional bacterial suspensions. Our measurements are free from effect of frictional walls and thus allow quantitative comparison with previous theoretical and computational works. We also present a detailed analysis on the flow fields generated by the swimming bacteria, and reveal a strong coupling between flow strength and giant number fluctuations spanning all length scales.

In addition, we measure the critical conditions of the transition from disordered state to turbulent state in bacterial suspensions. We present the experimental results in a phase diagram, serving as a benchmark for existing and future theories. We put forward a heuristic model based on two-body hydrodynamic interactions, hoping to understand the transition in a more intuitive way and to stimulate theoretical advancement.

By elucidating the causes and consequences these phenomena, we can not only expand the knowledge of active fluids, but also provide deeper understandings on the biological and ecological impact of living organism behavior. Our experiments allow quantitative understanding of active fluids and lay the foundation of applying active fluids to real world challenges.
\end{document}
